\documentclass[11pt]{article}
\title{\textbf{David's Role-Playing System\\
Core Rules}}
\author{Designer: David J. Setser\\
Editor: Dharma Bellamkonda}
\date{}
\usepackage{multirow}
\usepackage[bookmarks]{hyperref}
\begin{document}
\maketitle

\section{Introduction}

This document describes a system for tabletop role-playing games designed to be adaptable to any setting.

\section{Skills}
In this role-playing system, all actions (including combat actions) are based on \textbf{skills}.
\footnote{The list of skills available to characters is defined by the setting rules and the game master.}
A skill is a quantitative measure of how well a character can perform tasks that require the use of that skill.
Each skill consists of an associated \textbf{ability}, an \textbf{education bonus}, a \textbf{talent bonus}, any \textbf{conditional modifiers} and any \textbf{trait modifiers}.

When a character uses a skill to attempt a task, the controlling player (or game master) performs a \textbf{skill check} to determine whether the character succeeds or fails.
The player rolls two six-sided dice and adds their character's skill bonuses and modifiers to the outcome of the dice roll.
\footnote{Critical successes and/or critical failures can be optionally implemented by the setting rules or game master.
A critical success can be represented by applying a +6 conditional modifier when a player rolls "sixes" (two sixes).
A critical failure can be represented by applying a -6 conditional modifier when a player rolls "snake eyes" (two ones).}
The character succeeds if the total skill check is higher than the skill check target's \textbf{opposed skill check} or the skill check's \textbf{difficulty check}.
\footnote{Players always win ties against non-player characters or difficulty checks.} Otherwise, the character fails the task.

Opposed skill checks are used when the use of a skill involves competing against or targeting an animate character that is non-compliant or resisting.
The target rolls their own skill check of an appropriate action.
If the offensive character's total skill check is higher than the defensive character's total skill check, the offensive character succeeds.
If the defensive character's total skill check is higher than the offensive character's total skill check, the offensive character fails.

Difficulty checks are used whenever opposed rolls are not applicable to the situation.
The game master decides on a number that represents how difficult the skill check should be.
If the player's total skill check is higher than the difficult check, the character succeeds.
Otherwise, the character fails the task.

\subsection{Ability}
\textbf{Ability bonuses} are six quantitative scores that describe a character's natural abilities in broad areas.
Each ability is associated with several related skills. An ability bonus of 0 is an average human, while an ability bonus of 6 is highly exceptional.
Ability bonuses may be negative, indicating below-average ability in that area.\
footnote{The ability bonus distribution for player characters and the associated ability for each skill is defined by the setting rules and the game master.}

\begin{description}
	\item[Strength]Muscular strength.
	\item[Dexterity]Agility, reflexes, balance, and fine motor skills.
	\item[Constitution]Physical health and hardiness.
	\item[Intelligence]Learning and reasoning skills.
	\item[Wisdom]Awareness, intuition and common sense.
	\item[Charisma]Personality and ability to influence others.
\end{description}

\subsection{Education}
Each skill has an \textbf{education bonus}.
This is a quantitative measure of a character's formal training and study in a particular skill.
Education bonuses range from 0 to 12.

\begin{center}
	\begin{tabular}{cll}
		Bonus & Skill Level & Real World Equivalent  \\
		\hline
		0     & Unskilled   & No Education           \\
		2     & Beginner    & Some School            \\
		4     & Novice      & High School            \\
		6     & Apprentice  & Associate's Degree     \\
		8     & Master      & Bachelor's Degree      \\
		10    & Specialist  & Master's Degree        \\
		12    & Expert      & Doctoral Degree        \\
	\end{tabular}
\end{center}

Education bonuses are defined at character creation based on the character's previous training and education.
It is up to the individual setting rules and the game master to determine reasonable education bonuses for characters.
Education bonuses generally do not change after character creation unless the character undergoes additional education.

\subsection{Talent}
Each skill has a \textbf{talent bonus}
This is a quantitative measure of a character's experience and practice at a skill
Talents can be increased by spending \textbf{talent points}, which are the primary means by which characters increase in power.
Talent points are rewarded by the game master at game milestones such as completing a quest, advancing the story, or simply at the end of each play session.

Players may immediately spend a talent point to increase a single skill's talent bonus by 1.
The skill must have been used by the character at least once since the last time a talent point was spent, and a player can not increase the same skill's talent bonus consecutively (i.e twice or more in a row).

\subsection{Conditional Modifier}
The circumstances of a situation may have a positive or negative effect on your character's skills.
These are quantitatively represented by \textbf{conditional modifiers}.
\footnote{Conditional modifiers are defined and applied by the setting rules and the game master.}
Examples of conditional modifiers include:

\begin{itemize}
	\item A character wears set of heavy armor, applying a +2 conditional modifier to their defenses against weapons but also applying a -4 conditional modifier to skills involving physical exertion due to deceased mobility and extra weight.
	\item A character fires an arrow from atop a wall at an enemy below, applying a +2 conditional modifier to the attack due to the height difference.
	\item A character fails an opposed roll against another character's intimidation, applying a -1 conditional modifier to all actions on their next turn due to shaken nerves.
	\item A player thinks of an unexpected solution to a problem, granting a +2 conditional modifier to their attempt to execute the solution as a reward for player creativity.
\end{itemize}

\subsection{Traits}
\textbf{Traits} represent special or unusual ways a character uses their skills.
\footnote{Traits are defined by the setting rules and the game master and are assigned to characters by the game master based on character backgrounds and development.}
Traits are rarely purely negative, but may include both positive and negative aspects.
Examples of traits include:

\begin{description}
	\item[Polar Explorer]You spent time exploring a polar region and are experienced at surviving in cold conditions. Gain a +2 trait modifier to Survival checks against cold conditions.
	\item[Tinkerer]As a hobbyist you love seeing how things work. You may dissemble a mechanical object to understand its internal functions. However, you must pass a Crafting check to correctly reassemble the item.
	\item[Photosensitive]Your eyes are extra sensitive to light. Gain a +2 trait modifier to perception checks in low-light and dark environments, but gain a -1 trait modifier in bright environments.
	\item[Optimism]You try to make the best out of terrible situations. If any non-combat skill check results in a failure, you may choose to repeat the roll. You must accept the new result.
\end{description}

\section{Combat}
For combat and other similar situations, an additional set of rules is implemented to manage the order and timing of actions and defeat conditions.

\subsection{Condition}
A character's ability to function and act is represented by \textbf{condition}.
The core rules specify two condition types, \textbf{health} and \textbf{fatigue}.
Healthy characters start at the normal level of their condition.
As a character becomes progressively more injured or tired in a fight, their appropriate condition level is reduced and they incur negative conditional modifiers to all skill checks.
If any condition is reduced to the lowest level, the character becomes unable to act.
Characters can reverse a reduced condition level by finding in-game means of healing and rest.

The relative power tier of different characters is represented by the number of levels on their condition tracks.
More powerful characters have more condition levels and are more difficult to defeat.
\footnote{Setting player characters at the Heroic tier is recommended, but they may be set at any tier appropriate to the setting and intended difficulty.}

\begin{center}
	\begin{tabular}{|c|ll|ll|}
		\multicolumn{1}{c}{Tier} & \multicolumn{2}{c}{Health} & \multicolumn{2}{c}{Fatigue}\\
		\hline
		\multirow{8}{*}{Epic}
		 & Normal   & No Penalty              & Normal        & No Penalty              \\
		 & Bruised  & -1 Conditional Modifier & Drained       & -1 Conditional Modifier \\
		 & Battered & -2 Conditional Modifier & Weary         & -2 Conditional Modifier \\
		 & Beaten   & -2 Conditional Modifier & Fatigued      & -2 Conditional Modifier \\
		 & Injured  & -4 Conditional Modifier & Weakened      & -4 Conditional Modifier \\
		 & Wounded  & -4 Conditional Modifier & Exhausted     & -4 Conditional Modifier \\
		 & Bloodied & -8 Conditional Modifier & Depleted      & -8 Conditional Modifier \\
		 & Dying    & Cannot Act              & Incapacitated & Cannot Act              \\
		\hline
		\multirow{7}{*}{Legendary}
		 & Normal   & No Penalty              & Normal        & No Penalty              \\
		 & Bruised  & -1 Conditional Modifier & Drained       & -1 Conditional Modifier \\
		 & Beaten   & -2 Conditional Modifier & Fatigued      & -2 Conditional Modifier \\
		 & Injured  & -4 Conditional Modifier & Weakened      & -4 Conditional Modifier      \\
		 & Wounded  & -4 Conditional Modifier & Exhausted     & -4 Conditional Modifier \\
		 & Bloodied & -8 Conditional Modifier & Depleted      & -8 Conditional Modifier \\
		 & Dying    & Cannot Act              & Incapacitated & Cannot Act              \\
		\hline
		\multirow{6}{*}{Heroic}
		 & Normal   & No Penalty              & Normal        & No Penalty              \\
		 & Bruised  & -1 Conditional Modifier & Drained       & -1 Conditional Modifier \\
		 & Beaten   & -2 Conditional Modifier & Fatigued      & -2 Conditional Modifier \\
		 & Wounded  & -4 Conditional Modifier & Exhausted     & -4 Conditional Modifier \\
		 & Bloodied & -8 Conditional Modifier & Depleted      & -8 Conditional Modifier \\
		 & Dying    & Cannot Act              & Incapacitated & Cannot Act              \\
		\hline
		\multirow{5}{*}{Elite}
		 & Normal   & No Penalty              & Normal        & No Penalty              \\
		 & Beaten   & -2 Conditional Modifier & Fatigued      & -2 Conditional Modifier \\
		 & Wounded  & -4 Conditional Modifier & Exhausted     & -4 Conditional Modifier \\
		 & Bloodied & -8 Conditional Modifier & Depleted      & -8 Conditional Modifier \\
		 & Dying    & Cannot Act              & Incapacitated & Cannot Act              \\
		\hline
		\multirow{4}{*}{Veteran}
		 & Normal   & No Penalty              & Normal        & No Penalty              \\
		 & Wounded  & -4 Conditional Modifier & Exhausted     & -4 Conditional Modifier \\
		 & Bloodied & -8 Conditional Modifier & Depleted      & -8 Conditional Modifier \\
		 & Dying    & Cannot Act              & Incapacitated & Cannot Act              \\
		\hline
		\multirow{3}{*}{Common}
		 & Normal   & No Penalty              & Normal        & No Penalty              \\
		 & Wounded  & -4 Conditional Modifier & Exhausted     & -4 Conditional Modifier \\
		 & Dying    & Cannot Act              & Incapacitated & Cannot Act              \\
		\hline
		\multirow{2}{*}{Weak}
		 & Normal   & No Penalty              & Normal        & No Penalty              \\
		 & Dying    & Cannot Act              & Incapacitated & Cannot Act              \\
		\hline
	\end{tabular}
\end{center}

\subsection{Advantage}
\textbf{Advantage} is a relationship between two characters in combat where one is in a temporary superior state to the other.
Advantage is a one-way relationship; if a player gains advantage over a character, that character loses any advantage they had against that player.
A character may gain advantage over another in one of three ways.

\begin{itemize}
	\item The character successfully attacked the other and the other has not successfully attacked the character since then.
	\item At the start of their phase or action, the character is in a situation that grants advantage over the other (such as flanking the other or being on higher ground).
	\item The character performs a critical defense, immediately granting advantage to the defending character.
\end{itemize}

Advantage is a prerequisite condition for damaging an opponent and is carried over between phases.

\subsection{Phase and Action Order}
Combat is divided into \textbf{phases} during which all members of a side each perform \textbf{actions}.
Typically, this means that all players perform their actions during the players' phase, then all enemy characters perform their actions during the enemy phase.

Within a phase, the characters can perform their actions in any order they choose.
If the players do not have a preference for action order, they can simply take turns going around the table clockwise.

The side that acts first is determined by the role playing situation, such as which side takes the other by surprise.
If multiple sides start at the same time, the phase order can be decided by rolling one six-sided die for each side and going in order from highest roll to lowest.
Players always win ties against non-player characters.

\subsection{Actions}
An \textbf{action} is anything the character can do in approximately three seconds.
This might include attacking with a weapon, casting a spell, moving through the environment, or manipulating an object.
Actions are resolved with skill checks just as in non-combat situations.

During their side's phase, each character may choose to perform at least up to two sequential actions.
They may choose to take more actions by \textbf{multitasking} and/or \textbf{rushing}.
Multitasking allows a character to perform two simultaneous actions in place of a single action.
Multitasking applies a negative conditional modifier to both actions.
Rushing allows a character to perform three actions during a single phase with a negative conditional modifier on all three actions.
Both multitasking and rushing increase the consequences of failing an action.
Multitasking and rushing can be combined, but the negative conditional modifiers stack at an increasing rate.

\subsubsection{Movement}
The following table is a guideline of how far a character can move in a single action.

\begin{center}
	\begin{tabular}{lccc}
		Speed  & Feet & Squares & Hexes \\
		\hline
		Walk   & 30   & 6       & 6     \\
		Run    & 60   & 12      & 12    \\
		Sprint & 90   & 18      & 18    \\
	\end{tabular}
\end{center}

This speeds listed in the table can be modified by traits or conditional modifiers.
For example, a character carrying a heavy item may move more slowly, while a character who goes running every day may have a trait that allows them to move faster.

Different movement speeds apply different conditional modifiers to skills used later in the phase at the game master's discretion.
For example, sprinting before pushing a heavy object might incur a negative conditional modifier on the push action, while sprinting before jumping across a gap might grant a positive conditional modifier on the jump action.
Repeatedly sprinting might wear a character down, applying a negative conditional modifier on all their actions until they stop to rest.

\subsubsection{Attacking and Defending}
When a character attacks another character (whether it be with melee weapon, ranged weapon or magic-like ability) they use the associated attack skill to make an opposed skill check against their target.
The target uses an appropriate defense skill to oppose.
If the attack skill check succeeds, one of two results may occur.

\begin{itemize}
	\item If the attacker does not already have advantage over the defender, they gain advantage.
	\item If the attacker has advantage over the defender, the defender reduces the appropriate  condition track by one.
\end{itemize}

When the difference between the attacker's attack skill check and the defender's defense skill check is greater than 10 in favor of either character, a \textbf{combat critical} occurs.
The results of a combat critical are in addition to any normal results of the outcome.

\begin{itemize}
	\item If the combat critical is in favor of the attacker, they perform a \textbf{critical attack} and the defender goes down one more level on the condition track.
	\item If the combat critical is in favor of the defender, they perform a \textbf{critical defense} and immediately gain advantage over the attacker.
\end{itemize}

An attacker can choose to move up to 5 feet (one square or one hex) as part of their attack action.

\subsubsection{Prepared Actions}
A character can choose to execute an action during another side's phase by declaring a \textbf{prepared action}.
A character must forfeit one of their actions during their own side's phase in order to prepare an action.
The character's controlling player must specifically state what action the character will perform.
They may then execute that action at any time before their next phase.

\end{document}